\documentclass[12pt]{article}
\usepackage{graphicx}
\usepackage{enumerate}
\usepackage{amsmath}
\usepackage{amssymb, bm}
% \usepackage{bibtex}

\usepackage{icml2016/icml2016}



 
\title{Denoising EEG Data with Particle Filtering Methods}
\author{Drew Kristensen\and Adel Nabli\and Antoine Chehire}
\begin{document}
\maketitle  


The project gives you the opportunity to study in greater depth some concepts of the course. The topic has to be linked with algorithms, concepts or methods presented in class, but beyond this requirement, the choice is quite open. In particular, it may be tailored to your interests. We encourage you to choose a paper that closely fits your interests, and any personal original contribution is valued.

The standard class projects need to contain the following 3 components:

An article review around a given topic (research articles or chapter from Mike's book not studied in class). See below for a list of tentative projects. This means to read and understand a specific research article.

An implementation of the method.

An experimentation with real data. This means to apply the method on real data and report your findings and observations. If the paper is quite dense and theoretical, then an experimentation on simulated / synthetic data is sufficient.

A report (of about 4 to 8 pages) presenting the project and the obtained results (for applicative projects), to be given by December 19th, 2018 on Studium. The report has to be written in such a way that any student who has followed the class can understand (no need to introduce graphical model concepts). The report has to clearly present (in French or English) the studied problem and the existing approaches. You will be more evaluated on the clarity of the report rather than on its length. To train you to write professional research papers, you should use LaTeX in the ICML 2016 template format (download the template here). You may use appendices for additional details beyond 8 pages if you want, but be aware that as in standard conference reviewing, I might only read the first 8 pages (so the main content has to be there), and also, succinctness is more valued here than length!

\bibliographystyle{icml2016/isml2016}
\bibliography{sources}
\end{document}